\documentclass{article}
\usepackage[utf8]{inputenc}
\usepackage{titling}
\newcommand{\subtitle}[1]{%
  \posttitle{%
    \par\end{center}
    \begin{center}\large#1\end{center}
    \vskip0.5em}%
}

\title{Capytaine: waves and floating bodies}
\subtitle{CMSC6950 Course Project - 2021 Spring}
\author{Ruixin Song, 202093805}
\date{June 16, 2021}

\begin{document}

\maketitle

\section{Introduction}
Capytaine is a python-based package reproduces an open-sourced software Nemoh, which is a Boundary Element Method (BEM) solver in the Computational Fluid Dynamics (CFD) domain. Applied to python, Capytaine reinforces the solver by utilizing the computational flexibility in python and combining with other efficient tools in the python scientific ecosystem such as VTK, NumPy, xarray, etc. \\

More specifically, Capytaine focuses on solving the problems of the water waves interacting with the floating bodies. It supports the computation of added mass, radiation damping, diffraction force and Froude-Krylov force for rigid bodies. For the theoretical part, it can perform Green Function and Kochin Function calculation inherited from Nemoh's core functionalities. Aside from these, Capytaine implemented 3D animation of the floating bodies motion and free surface elevation, which can function with the Visualization Toolkit (VTK).\\

With the features in Capytaine, the project defines several computational problems and uses the functionalities in the package to solve them, then extracts subset data from the results and makes them visible with Matplotlib and VTK.

\section{Computational tasks}
what are the tasks?
functionalities/lib func calls is being used to address the tasks
data input

\section{Visualization}
\section{Workflow}
\section{bibliography}

\end{document}
